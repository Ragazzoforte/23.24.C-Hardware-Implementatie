\documentclass[openany]{book}


\usepackage{graphicx} % for including graphics
\graphicspath{{images/}}
\usepackage{amsmath}  % for advanced math formatting
\usepackage{hyperref}  % for hyperlinks
\usepackage{appendix}
\usepackage[table]{xcolor}
\usepackage{caption}
\usepackage{geometry}
\newgeometry{left=3.5cm,right=3.5cm,top=3cm,bottom=3cm}
\usepackage{multicol}
\usepackage{subcaption}


\begin{document}



\title{%
    % \begin{figure}
    %     \centering
    %     \includegraphics[scale=0.28]{Michel_nog bijsnijden.jpg}
    % \end{figure}
  Hardware Implementate \\
  \large PCB Design}
\author{Michel Vollmuller \\
        E-mail: michel.vollmuller@student.hu.nl \\
        Studentnummer: 1809572 \\
        Opleiding: Elektrotechniek HU}
\date{\today}

\maketitle

\chapter*{Inleiding}
\addcontentsline{toc}{chapter}{Inleiding}
Voor degene die nog niet weten wie ik ben. Mijn naam is Michel Vollmuller. Op het moment van schrijven ben ik 21 jaartjes jong en volg ik de studie Elektrotechniek aan de Hogeschool Utrecht. In de afbeelding hieronder zie je mij naast mijn vriendin in de achtbaan en die was nog best heftig, maar volgensmij was dat al wel zichtbaar aan de foto (diabolik in movie park italie). Verder heb ik nog als hobby's voetbal, wintersport, vissen, hobbyen en nog veel meer maarja dat is natuurlijk niet waar dit portfolio over gaat.\\
Tijdens het 3e leerjaar krijgen wij de mogelijkheid om ons tijdens het quest project professioneel vlak te ontwikkelen. Dit doen wij op basis van 3 competenties. Mijn 3 competenties waren;
\begin{itemize}
    \item Communiceren
    \item Reflecteren
    \item Samenwerken
\end{itemize}
Ik zal jullie een inzicht geven in mijn ontwikkelproces van deze 3 competities en in de bijlage zullen verschillende hulpmiddelen te vinden zijn met enkele bewijsstukken.

% \begin{figure}
% \centering
% \includegraphics[scale=0.1]{20230810_190215.jpg}
% \end{figure}

\tableofcontents

\chapter{Opdracht}
    \input{Opdracht.tex}

\chapter{Schema}
    \input{Schema.tex}

\chapter{PCB}
    \input{PCB.tex} 

\chapter{Conclusie}
    \input{conclusie.tex} 

% Bijlage's    
\appendix



\end{document}